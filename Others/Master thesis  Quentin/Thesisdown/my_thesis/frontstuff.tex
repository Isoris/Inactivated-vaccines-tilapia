%==========================================================
%======================  FRONT STUFF ======================
%==========================================================

%======================= Cover Page =======================
%\begin{document}
\newpage
\pagestyle{plain}
\pagenumbering{roman}
\setcounter{page}{0}
\setcounter{tocdepth}{3}
%\setlength{\footskip}{-15mm}
\setlength{\textheight}{230mm}
\setlength{\voffset}{10mm}
\vskip -5em

\begin{center}
{ 
  \singlespace \uppercase{\bf SYSTEMIC AND MUCOSAL IMMUNE RESPONSE OF NILE TILAPIA BROODSTOCK TO MONOVALENT AND BIVALENT VACCINES AGAINST BACTERIA STREPTOCOCCUS AGALACTIAE AND AEROMONAS VERONII} \par
}
\vskip 2em
{
  \lineskip 1.5em
  \begin{tabular}[t]{c} by\\ \\ QUENTIN ANDRES
  \end{tabular}\par
}
%\vskip 5.6em

\vskip 3.5em
\singlespace A thesis submitted in partial fulfillment of the
requirements for the degree of Master of Science in Aquaculture and Aquatic Resources Management

%	\vskip 5em
\vskip 0.5em
{
  \singlespace
  \begin{center}
    \begin{tabular}{rl}
      \\[-1em]
      Examination Committee: & Dr.\ Krishna R.Salin (Chairperson) \\[-0.8em]
                             & Dr.\ Ha Thanh Dong \#1 \\[-0.8em]
                             & Dr.\ YOUR COMMITTEE \#2 \\\\

% UNCOMMENT THE LINES BELOW IF YOU HAVE THE EXTERNAL EXAMINER.
%      External Examiner:     & Prof.\ YOUR EXTERNAL EXAMINER \\[-0.8em]
%                             & Dept.\ of Electrical and Computer Engineering \\[-0.8em]
%                             & McGill University, Canada \\\\
		
      Nationality:     & French \\[-0.8em]
      Previous Degree: & Bachelor of Molecular Biology, Microbiology
                             and Genetics
\\[-0.8em]
                       & 	University of Nice – Côte d’Azur
		Nice, France\\[-0.8em]
      \\
      Scholarship Donor: & None\\[-0.8em]
      \\
    \end{tabular}
  \end{center}

  \vskip 3.0em
  \centerline{}
  \vskip 2em
}
\end{center}
\begin{center}
  \singlespace Asian Institute of Technology\\ School of Environment, Resources and Development\\ Thailand\\ April 2022
\end{center}
\vfill

%====================== AUTHOR’S DECLARATION ======================
\newpage
\pagestyle{plain}
\onecolumn % Single-column.
\if@twoside\else\raggedbottom\fi % Ragged bottom unless twoside option.
\setlength{\footskip}{8mm}
\begin{center}
{
  \large \bf AUTHOR’S DECLARATION\\ \vskip 1em
}
\vskip 1em
\end{center}
\singlespace
\doublespace
\hspace{8.5mm}
\vspace{-1em}

I, Quentin ANDRES, declare that the research work carried out for this thesis was in accordance with the regulations of the Asian Institute of Technology. The work presented in it are my own and has been generated by me as the result of my own original research, and if external sources were used, such sources have been cited. It is original and has not been submitted to any other institution to obtain another degree or qualification. This is a true copy of the thesis, including final revisions.
\begin{center}
\vskip 0.5em
{
  \singlespace
\begin{center}
    \begin{tabular}{rl}
      \\[-1em]

\\Date: &  \\[-0.8em]

\\Name (in printed letters): & QUENTIN ANDRES\\[-0.8em]

\\Signature: & \\[-0.8em]

      \\
    \end{tabular}
  \end{center}
  \vskip 3.0em
  \centerline{}
  \vskip 2em
}
\end{center}

%====================== ACKNOWLEDGEMENTS ======================

\newpage
\pagestyle{plain}
\onecolumn % Single-column.
\if@twoside\else\raggedbottom\fi % Ragged bottom unless twoside option.
\setlength{\footskip}{8mm}
\begin{center}
{
  \large \bf ACKNOWLEDGMENTS\\ \vskip 1em
}
\vskip 1em
\end{center}
\singlespace
\doublespace
\hspace{8.5mm}
\vspace{-1em}

Type your acknowledgments here. This section is typically reserved for personal and professional dedications. You may also acknowledge your donor or funder here or include copyright acknowledgements of journals whose articles you have significantly incorporated in this thesis. Maintain one blank 1.5 line spacing between paragraphs.


[one page, maximum]
\phantomsection
\addcontentsline{toc}{section}{ACKNOWLEDGMENTS}

%====================== ABSTRACT ======================
\newpage
\pagestyle{plain}
\begingroup
\hypersetup{linkcolor=black}
% \renewcommand\listfigurename{LIST OF FIGURES}


\phantomsection
\addcontentsline{toc}{section}{ABSTRACT}
\onecolumn % Single-column.
\if@twoside\else\raggedbottom\fi % Ragged bottom unless twoside option.

\setlength{\footskip}{8mm}

\begin{center}
{\large \bf ABSTRACT \\ \vskip 1em}
\vskip 1em
\end{center}
\singlespace
\doublespace
\hspace{8.5mm}
\vspace{-1em}

Fighting bacterial infections inducing mass mortality in fish is a hot-topic research in the aquaculture industry in order to sustain its intensification. This research project aims to develop monovalent and bivalent vaccines against \ac{gbs} \acs{s.agalac} and against \ac{gp} \ac{a.veronii} for prophylaxis of \underline{\textit{Nile tilapia}} (\acs{onilo}) for which there is currently no vaccine available in Thailand.  In the present research, \acs{fkv} were produced by pathogen inactivation method: formalin 1\% - 3 \acs{(v/v)} is added to the virulent pathogenic solution. The objective of the study is to characterize a part of the immune response elicited in \aca{onilo} in response to vaccination. A total of $(\mathrm{n} = 400 juveniles)$ with an average mass of $(\mathrm{m} = X\si{\gram} \mypm X'\si{\gram})$ were stocked into experimental ponds/aquariums prior to the start of the experiments. The population was then divided into 4 groups according to experimental design. After verification that the fish were free of any disease, they were acclimated for about a week. Meanwhile our 2 virulent pathogenic bacteria have been recovered and amplified on appropriate medium and were used for the vaccine production but also stored for challenge test. As a first step, a total of 4 treatments were administrated to the fish in order to monitor a possible immunization: 1 \Ac{control} (a vaccine with no immunizing properties) & 3 formalin inactivated vaccines \ac{fkv} containing either antigenic particles of \ac{s.agalac} (monovalent \acs{sa}); \ac{a.veronii} (monovalent \acs{av}); \ac{s.agalac} + \ac{a.veronii} (bivalent \acs{saav} with a 1:1 ratio). In a second step, the fish immune response to vaccination (=immunogenicity and survivability) was determined by titrating agglutinating antibodies \Acs{ab} but also with \acs{elisa} assay which is more reliable method. Indeed, isotype determination of \Acs{ab} can allow to determine the concentrations of specific \ac{igm} and \ac{igt} for \textit{\ac{s.agalac}'s} polysaccharides and \ac{a.veronii}. In addition, gene transcription activity from before, during, and after the immunization (up to 70 \ac{dpv}) were determined by \acs{rtpcr} for \ac{igm} and \ac{igt} (active transcripts) corresponding to humoral immunity, and for \ac{sod}, \ac{cat}, \ac{gpx}, \ac{nos} (respiratory burst activity of phagocytes, \ac{ros} producing) corresponding to innate immunity. As a final step, a challenge test in vivo was carried out on the fish juveniles with the two bacteria. Fish were injected with \acf{control}+\ac{pbs} or were \aca{ip} a lethal dose of pathogen according to their prior vaccination.

\endgroup
\setlength{\parskip}{0pt} 
%======================= Table of Contents =========================
\newpage
\phantomsection
\tableofcontents
% Page 2 begins
% %===================== List of Tables ======================
\newpage
\phantomsection
\addcontentsline{toc}{section}{LIST OF TABLES}
\listoftables

%===================== List of Figures ======================
\newpage
\phantomsection
\addcontentsline{toc}{section}{LIST OF FIGURES}
\listoffigures
\phantomsection
%\clearpage     % \clearpage ends the page, and also dumps out all floats.
%\end{document} % Floats are things like tables and figures.

%\clearpage     % \clearpage ends the page, and also dumps out all floats.

% %\clearpage     % \clearpage ends the page, and also dumps out all floats.
% %\end{document} % Floats are things like tables and figures.

%\clearpage     % \clearpage ends the page, and also dumps out all floats.

\setlength{\parskip}{12pt}

\newpage
\phantomsection
\addcontentsline{toc}{section}{LIST OF ABREVIATIONS}
\printacronyms[name= {\large LIST OF ABREVIATIONS}]
